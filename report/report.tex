% Options for packages loaded elsewhere
% Options for packages loaded elsewhere
\PassOptionsToPackage{unicode}{hyperref}
\PassOptionsToPackage{hyphens}{url}
\PassOptionsToPackage{dvipsnames,svgnames,x11names}{xcolor}
%
\documentclass[
]{article}
\usepackage{xcolor}
\usepackage[margin=1in]{geometry}
\usepackage{amsmath,amssymb}
\setcounter{secnumdepth}{5}
\usepackage{iftex}
\ifPDFTeX
  \usepackage[T1]{fontenc}
  \usepackage[utf8]{inputenc}
  \usepackage{textcomp} % provide euro and other symbols
\else % if luatex or xetex
  \usepackage{unicode-math} % this also loads fontspec
  \defaultfontfeatures{Scale=MatchLowercase}
  \defaultfontfeatures[\rmfamily]{Ligatures=TeX,Scale=1}
\fi
\usepackage{lmodern}
\ifPDFTeX\else
  % xetex/luatex font selection
\fi
% Use upquote if available, for straight quotes in verbatim environments
\IfFileExists{upquote.sty}{\usepackage{upquote}}{}
\IfFileExists{microtype.sty}{% use microtype if available
  \usepackage[]{microtype}
  \UseMicrotypeSet[protrusion]{basicmath} % disable protrusion for tt fonts
}{}
\makeatletter
\@ifundefined{KOMAClassName}{% if non-KOMA class
  \IfFileExists{parskip.sty}{%
    \usepackage{parskip}
  }{% else
    \setlength{\parindent}{0pt}
    \setlength{\parskip}{6pt plus 2pt minus 1pt}}
}{% if KOMA class
  \KOMAoptions{parskip=half}}
\makeatother
% Make \paragraph and \subparagraph free-standing
\makeatletter
\ifx\paragraph\undefined\else
  \let\oldparagraph\paragraph
  \renewcommand{\paragraph}{
    \@ifstar
      \xxxParagraphStar
      \xxxParagraphNoStar
  }
  \newcommand{\xxxParagraphStar}[1]{\oldparagraph*{#1}\mbox{}}
  \newcommand{\xxxParagraphNoStar}[1]{\oldparagraph{#1}\mbox{}}
\fi
\ifx\subparagraph\undefined\else
  \let\oldsubparagraph\subparagraph
  \renewcommand{\subparagraph}{
    \@ifstar
      \xxxSubParagraphStar
      \xxxSubParagraphNoStar
  }
  \newcommand{\xxxSubParagraphStar}[1]{\oldsubparagraph*{#1}\mbox{}}
  \newcommand{\xxxSubParagraphNoStar}[1]{\oldsubparagraph{#1}\mbox{}}
\fi
\makeatother


\usepackage{longtable,booktabs,array}
\usepackage{calc} % for calculating minipage widths
% Correct order of tables after \paragraph or \subparagraph
\usepackage{etoolbox}
\makeatletter
\patchcmd\longtable{\par}{\if@noskipsec\mbox{}\fi\par}{}{}
\makeatother
% Allow footnotes in longtable head/foot
\IfFileExists{footnotehyper.sty}{\usepackage{footnotehyper}}{\usepackage{footnote}}
\makesavenoteenv{longtable}
\usepackage{graphicx}
\makeatletter
\newsavebox\pandoc@box
\newcommand*\pandocbounded[1]{% scales image to fit in text height/width
  \sbox\pandoc@box{#1}%
  \Gscale@div\@tempa{\textheight}{\dimexpr\ht\pandoc@box+\dp\pandoc@box\relax}%
  \Gscale@div\@tempb{\linewidth}{\wd\pandoc@box}%
  \ifdim\@tempb\p@<\@tempa\p@\let\@tempa\@tempb\fi% select the smaller of both
  \ifdim\@tempa\p@<\p@\scalebox{\@tempa}{\usebox\pandoc@box}%
  \else\usebox{\pandoc@box}%
  \fi%
}
% Set default figure placement to htbp
\def\fps@figure{htbp}
\makeatother





\setlength{\emergencystretch}{3em} % prevent overfull lines

\providecommand{\tightlist}{%
  \setlength{\itemsep}{0pt}\setlength{\parskip}{0pt}}



 


\makeatletter
\@ifpackageloaded{caption}{}{\usepackage{caption}}
\AtBeginDocument{%
\ifdefined\contentsname
  \renewcommand*\contentsname{Table of contents}
\else
  \newcommand\contentsname{Table of contents}
\fi
\ifdefined\listfigurename
  \renewcommand*\listfigurename{List of Figures}
\else
  \newcommand\listfigurename{List of Figures}
\fi
\ifdefined\listtablename
  \renewcommand*\listtablename{List of Tables}
\else
  \newcommand\listtablename{List of Tables}
\fi
\ifdefined\figurename
  \renewcommand*\figurename{Figure}
\else
  \newcommand\figurename{Figure}
\fi
\ifdefined\tablename
  \renewcommand*\tablename{Table}
\else
  \newcommand\tablename{Table}
\fi
}
\@ifpackageloaded{float}{}{\usepackage{float}}
\floatstyle{ruled}
\@ifundefined{c@chapter}{\newfloat{codelisting}{h}{lop}}{\newfloat{codelisting}{h}{lop}[chapter]}
\floatname{codelisting}{Listing}
\newcommand*\listoflistings{\listof{codelisting}{List of Listings}}
\makeatother
\makeatletter
\makeatother
\makeatletter
\@ifpackageloaded{caption}{}{\usepackage{caption}}
\@ifpackageloaded{subcaption}{}{\usepackage{subcaption}}
\makeatother
\usepackage{bookmark}
\IfFileExists{xurl.sty}{\usepackage{xurl}}{} % add URL line breaks if available
\urlstyle{same}
\hypersetup{
  pdftitle={COVID-19 Mortality Analysis Report},
  pdfauthor={Dane Pearson; Dhriti Avala},
  colorlinks=true,
  linkcolor={blue},
  filecolor={Maroon},
  citecolor={Blue},
  urlcolor={Blue},
  pdfcreator={LaTeX via pandoc}}


\title{COVID-19 Mortality Analysis Report}
\usepackage{etoolbox}
\makeatletter
\providecommand{\subtitle}[1]{% add subtitle to \maketitle
  \apptocmd{\@title}{\par {\large #1 \par}}{}{}
}
\makeatother
\subtitle{A Statistical Investigation using R}
\author{Dane Pearson \and Dhriti Avala}
\date{2025-12-09}
\begin{document}
\maketitle


\begin{center}
\textbf{\Large Introduction}
\end{center}

~~~~The COVID-19 pandemic has had dramatic and lasting effect on the
United States, specifically in public health, society, and policy.
Between the years of 2020 and 2023, over one million Americans died due
to COVID-19, with mortality patterns varying based on demographic
groups, regions, and seasonality. Being able to understand these
patterns and how specific characterisitcs affected mortality rates is
important in characterizing the impact of the pandemic and evaluating
public health implementations to combat the virus. The purpose of this
study is to analyze these mortality patterns utilizing provisional death
counts provided by the Center of Disease Control (CDC) in addition to
state level population estimations.

~~~~ With the help of prior research of the CDC and National Center of
Health Statistics (NCHS), we know that the older generation seemed to
have experienced significantly higher death counts, frequency increasing
with age. Additionally, studied have shown that there is a higher
mortality rate in men than women, potentially hinting at differences in
comorbidity, immune response, and behavioral factors. Geographic
variation has also been reported, where the South and Midwest seemed to
have experienced mortality rates, potentially due to differences in
vaccination rates, population density, healthcare policies, and the
timeline of policy implementation.

~~~~The primary dataset used in this study is ``Provisional COVID-19
Deaths by Sex and Age'', (CDC, National Center for Health Statistics,
2020). This file contains key variables such as monthly and yearly death
count for each U.S state from 2020 to 2023, broken down further with
supporting variables such as sex and age group. The dataset contains
over 138,000 observations and CDC-supressed values (NA values) in low
population subgroups in efforts to preserve confidentiality and avoid
systematic bias. In order to estimate comparalbe mortality rates across
the states and demographics, the data was left-joined with 2020-2023
state population estimates from the U.S Census (US Census Beureau, 2024)
on the ``State'' variable. By joining the datasets, we were able to
calculate death rates of each state in different months and years. In
addition to this, states were also grouped into four regions: West,
MidWest, South, and Northeast to help us perform by-region analysis.

~~~~Our study aimed to focus on four main objectives. First, we aimed to
analyze the patters in U.S COVID-19 using detailed analysis and
graphical visualizations of death rates across all months from 2020 to
2023. Second, we aimed to examine demographic patterns by analyzing
deaths across age groups and sex. Third, we used visualizations such as
heatmaps to depict state-level variation in COVID-19 death rates.
Lastly, we employed a variety of modeling approaches such as Multiple
Linear Regression, Poisson Regression, Negatve Binomial Regression, and
Two-Way ANOVA to quantify the effects of year, region, age group, and
sex on COVID-19 mortality. Ultimately, these analyses helped to provide
a comprehensive understanding of factors associated with COVID-19 death
rates in the United States and revealed how mortality has changed over
the course of the pandemic.

\textbf{\large Exploratory Data Analysis}

\begin{center}
\text{Table 1. Summary of Provisional COVID-19 Deaths}
\end{center}

\begin{longtable}[]{@{}
  >{\raggedright\arraybackslash}p{(\linewidth - 2\tabcolsep) * \real{0.2432}}
  >{\raggedright\arraybackslash}p{(\linewidth - 2\tabcolsep) * \real{0.7568}}@{}}
\toprule\noalign{}
\begin{minipage}[b]{\linewidth}\raggedright
\textbf{Variable}
\end{minipage} & \begin{minipage}[b]{\linewidth}\raggedright
\textbf{Categories}
\end{minipage} \\
\midrule\noalign{}
\endhead
\bottomrule\noalign{}
\endlastfoot
Group & By Month, By Year, By Total \\
Sex & All Sexes, Female, Male \\
Age Group & 0--17 years, 1--4 years, 15--24 years, 18--29 years, 25--34
years, 30--39 years, (Other) \\
\end{longtable}

\begin{center}
\text{Table 2. Summary of State Populations Data}
\end{center}

\begin{longtable}[]{@{}llll@{}}
\toprule\noalign{}
\textbf{Variable} & \textbf{Min} & \textbf{Median} & \textbf{Max} \\
\midrule\noalign{}
\endhead
\bottomrule\noalign{}
\endlastfoot
Pop 2020 & 577664 & 6025563 & 331526933 \\
Pop 2021 & 579548 & 6025186 & 332048977 \\
Pop 2022 & 581629 & 6027262 & 333271411 \\
Pop 2023 & 584057 & 6045604 & 334914895 \\
\end{longtable}

~~~~Our original, unaltered ``Provisional COVID-19 Deaths by Sex and
Age'' (CDC, National Center for Health Statistics, 2020) dataset from
the CDC has 138,000 rows and 16 columns. The key columns we focused on
for this project are ``Group'' (whether data was measured by Month, by
Year, or Total), ``Year'', ``Month'', ``State'', ``Sex'', ``Age Group'',
and ``COVID.19.Deaths''. Before conducting the main exploratory
analysis, we examined the number of missing values (NA) in each of the
variables of interest. We did not count the number of missing values of
the ``Year'' and ``Month'' columns, as they have missing values
depending on whether the respective row was measured by year or by
month. Each key column had no missing values, aside from
``COVID.19.Deaths'', which had 39,430 rows with no value.

~~~~Following this, we explored how the frequency of deaths changed over
time in the data, plotting the average total deaths across in the entire
US for each month, from 2020-2023. We found that deaths from COVID-19
spiked drastically in late 2020 and early 2021, with the death rate
lowering in mid 2021 and leveling off by early 2023.

~~~~The decline COVID-19 deaths in mid-2021 coincided with the release
and administration of the vaccine, as more than 70\% of the U.S.
population had already received at least one vaccine dose, and
approximately 60\% were fully vaccinated by this time. These
developments may have contributed to the observed reduction in
mortality, as vaccines have been shown to substantially decrease the
risk of severe illness, hospitalization, and transmission (CDC, 2025).
Although this was insightful, we wanted to dive deeper by examining the
distribution of each month's COVID-19 deaths in the US, averaged across
2020-2023. By creating a boxplot for each month, a clear seasonal
pattern was revealed: mortality was highest in December and January and
lowest during the summer months, particularly June.

\begin{figure}

\begin{minipage}{0.50\linewidth}
\pandocbounded{\includegraphics[keepaspectratio]{../images/monthly_dists.png}}\end{minipage}%
%
\begin{minipage}{0.50\linewidth}
\pandocbounded{\includegraphics[keepaspectratio]{../images/deaths_time.png}}\end{minipage}%

\end{figure}%

~~~~This pattern indicates that higher COVID-19 mortality in the United
States was associated with colder periods of the year, whereas lower
mortality aligned with warmer periods. Seasonal factors may help explain
this, as colder and drier conditions can weaken immune defenses and aid
viruses in surviving longer. Increased time inside during colder months
may have also promoted greater transmission of the COVID-19 virus.

~~~~Next, we analyzed how COVID-19 mortality varied across the different
age groups. We found that older ages were directly associated with
increased deaths. Those age 85 and older accounted for 311,863 total
COVID-19 deaths across all states from 2020-2023 in the United States,
while individuals younger than 35 accounted for 16,735 COVID-19 deaths.

~~~~Finally, we examined how mortalities varied both temporally and
spatially in the United States by creating choropleth heatmaps of the
mortality rate of each state, for each year in the dataset. By joining
our original dataset with data on 2020-2023 state population estimates
from the U.S Census, we were able to calculate the COVID-19 deaths per
100,000 people in each state. Our heatmap revealed that in 2020 and
2021, mortality rates were higher and demonstrated substantial variation
across states, with states in the Midwest, South, and Northeast
exhibiting the highest rates. By 2022, overall mortality levels declined
across the United States, though southern states still maintained higher
mortality rates. In 2023, mortality rates were lower nationwide with
little-to-no variation between states.

\includegraphics[width=0.75\linewidth,height=\textheight,keepaspectratio]{../images/heatmap.png}

~~~~Preliminary patterns in the data indicated that year, region, age
group, and sex were important sources of variability in COVID-19
mortality. The substantial variation across states and demographic
groups drove us to use modeling approaches suited for count data and
population-standardization.

\begin{center}
\textbf{\Large Methods}
\end{center}

~~~~In order to understand how COVID-19 mortality varies across time,
demographics groups, and U.S regions, we used a combination of
exploratory visualization and statistical modeling techniques. Each
method was selected accordingly to adhere to the structure of the
dataset, count based, non-normally distributed outcomes with strong
demographic groupings. We conducted our analysis in R using the
following packages: tidyverse (Wickham et al., 2019) for data
manipulation and visualization, lubridate (Grolemund \& Wichhamm, 2011)
for date handling, ggplot (Wickham, 2016) for plotting, maps (Becker et
al, 2023) for state analysis, MASS (Venables \& Ripley, 2002) for
Negative Binomial modeling, patchwork (Pedersen et al., 2024) to combine
plots and knitr (Xie, Y., 2024) for embedding code. For formatting
tables we used flextable (Gohel, et al., 2024), modelsummary
(Arel-Bundock, V., 2022), and kableExtra (Zhu, H., 2024).

\textbf{Multiple Linear Regression (MLR)}

~~~~As a baseline model we fit a multiple linear regression predicting
standardized mortality rates:

\[
Y_i = \beta_0 + \beta_1 \text{Year}_i + \beta_2 \text{AgeGroup}_i +
\beta_3 \text{Sex}_i + \beta_4 \text{Region}_i + \varepsilon_i
\]

~~~~We included Year as a factor rather than a numeric variable because
early exploratory analysis showed a non-linear decline in mortality
rates over time. Mulitple Linear Regression provides estimates that
quantify differences between demographics and geographic subgroups.
Additionally, MLR assumes linearity between the predictors and outcomes,
error homoscedasticity, and normally distributed residuals. We developed
some diagnostic plots which indicated violations of linearity and
constant variance, suggesting that linear regression may provide a
biased standard error, pushing for the use of a count-based generalized
linear model.

\textbf{Poisson Regression}

~~~~Since COVID-19 death counts are non-negative integers, Poisson was
an appropriate next step for us. We modeled observed COVID-19 deaths as:

\[
Y_i \sim \text{Poisson}(\lambda_i), 
\qquad
\log(\lambda_i) = 
\beta_0 
+ \beta_1 \text{Year}_i 
+ \beta_2 \text{AgeGroup}_i 
+ \beta_3 \text{Sex}_i 
+ \beta_4 \text{Region}_i 
+ \log(\text{Population}_i)
\]

~~~~The log link ensures that the predicted count values are positive,
and the offset term log(Populationi) adjusts for differences in
population size so that expected deaths scale according to the state's
population. Poisson regression is appropriate when the variance and mean
of the count data are similar, however, COVID-19 mortality data often
has a higher variance than mean, leading to bias estimations and our
consideration for a model that is a bit more flexible.

\textbf{Negative Binomial Regression}

~~~~Due to overdispersion in mortality rate, we implemented a Negative
Binomial regression model using the MASS package. This model assumes:

\[
Y_i \sim \text{NegBin}(\mu_i, k), 
\qquad
\log(\mu_i) = X_i \beta + \log(\text{Population}_i),
\]

where (k) is the dispersion parameter. The Negative Binomial model
incorporates an additional variance term:

\[
\text{Var}(Y_i) = \mu_i + \frac{\mu_i^2}{k}
\]

making it a more flexible model in comparison to the Poisson model. This
model was able to provide a better fit and lower AIC values, proving to
be more reliable and a better predictor.

\textbf{Two-Way ANOVA and Interaction Effects}

~~~~To understand whether mortality differed based on age group, sex,
and region, as well as understanding how these factors interacted, we
conducted a Two-Way ANOVA. \[
Y_{ijk} = 
\mu 
+ \alpha_i(\text{AgeGroup}) 
+ \beta_j(\text{Sex}) 
+ \gamma_k(\text{Region}) 
+ (\alpha\beta)_{ij} 
+ (\alpha\gamma)_{ik} 
+ (\beta\gamma)_{jk} 
+ \varepsilon_{ijk}
\]

~~~~Although death count is discrete, our dependent variable was mean
deaths conditional on demographic subgroups, making ANOVA valid for
comparison of group level means. Our two-way ANOVA exhibited a
significant interaction between Age Group x Sex, supporting the idea
that mortality differences across age groups depends on sex, and vice
versa.

\textbf{Spatial and Panel Data Methods}

~~~~We merged the CDC dataset with the U.S state populations from the
maps package in order create a heatmap to visualize geographic variation
in mortality rates. In order to do this, we had to standardize state
names, convert populate estimates into long format, and compute yearly
death rates per 100,000 residents.

~~~~We also developed a region-level panel dataset by aggregating
monthly deaths and population counts across states within each region.
This allowed us to create a visualization displaying month-to-month
trends and regional differences. We also used ggplot2 to plot mortality
trajectories, dedicating a line to each region, allowing for clear
comparison and pattern analysis.

\textbf{Why These Methods}

~~~~Using a combination of linear regression, Poisson and Negative
Binomial linear models, ANOVA, and panel-data visualization allowed us
to conduct a thorough analysis of this multidimentional dataset. We
chose Poisson and Negative Binomial models since COVID-19 mortality is
count-based, high skewed, and overdispersed, something that classic
linear models would struggle to handle. ANOVA and visualizations helped
us to provide insightful subgroup comparisons. These methods aligned
with our goals of quantifying differences across demographic groups,
regions, and years, as well as indentifying which factors had strong
influences on COVID-19 mortality between 2020 and 2023.

\begin{center}
\text{\Large Results}
\end{center}

\textbf{Multiple linear Regression}

~~~~With an adjusted R² of 0.636, our multiple linear regression model
explained 63.6\% of the variance in COVID-19 mortality rates. The
results demonstrate that mortality did not follow a uniform national
pattern but instead shifted over time and varied across demographics.
Mortality rates rose above the 2020 baseline and reached their maximum
in 2021, then declined quickly in 2022 and 2023. Age was the strongest
predictor (both p \textless{} 0.001), as those aged 50-64 experienced
rates 7.17 per 100,000 higher than baseline, while those 85+ showed
rates 11.45 per 100,000 higher. States in the South had higher mortality
than those in the Midwest, while the Northeast and West experienced
significantly lower rates. This suggests potential regional disparities
in exposure or health policies. Finally, males had consistently higher
mortality than females.

\begin{figure}

\begin{minipage}{0.40\linewidth}
\pandocbounded{\includegraphics[keepaspectratio]{../images/MLR.png}}\end{minipage}%
%
\begin{minipage}{0.60\linewidth}
\pandocbounded{\includegraphics[keepaspectratio]{../images/diagnostic_plots.png}}\end{minipage}%

\end{figure}%

~~~~The Residuals vs Fitted plot demonstrates that our MLR model clearly
violates the asumption of linearity. The wavy LOESS curve demonstrates
an uneven spread around the residuals = 0 line, indicating that the
model is potentially overpredicting in some ranges and underpredicting
in others. The assumption of constant variance (homoscedasticity) is
also violated, exhibited by the widening funnel shape in the plot. This
means that the variance in error grows as the predicted mortality
increases. The Q-Q plot demonstrates that our MLR model also violates
the asumption of normality of the residuals. The data is light-tailed,
suggesting the data may be right-skewed and the model potentially
underestimating high death-rate observations.

\begin{figure}[H]

\begin{minipage}{0.50\linewidth}
\pandocbounded{\includegraphics[keepaspectratio]{../images/poisson.png}}\end{minipage}%
%
\begin{minipage}{0.50\linewidth}
\pandocbounded{\includegraphics[keepaspectratio]{../images/neg_bin.png}}\end{minipage}%

\end{figure}%

~~~~In our Poission model, males were predicted to have had about a 28\%
higher mortality rate than females, while regional effects showed that
the Northeast and West had 14\% and 16\% lower mortality compared to the
Midwest baseline. The South had about 10\% higher mortality than the
Midwest. Mortality declined over time, with each additional year
predicting a 35\% decrease in COVID-19 death rates, potentially being a
result of the adinistrtion of the vaccine. Age was the strongest
predictor, with risk increasing steadily across age groups and the 85+
population experiencing more than 150 times the mortality rate of
children, which may be explained by older people having weaker immune
systems.

~~~~In the Negative Binomial Regression model, males were predicted to
have had a 37\% higher mortality rates than females. Southern states
showed a 36\% higher mortality rate, while the Northeast experienced
about 24\% lower mortality compared to the Midwest. Mortality declined
substantially over time, with each additional year predicting a 44\%
decrease in COVID-19 death rates. This trend also might be explained by
the release of the COVID-19 vaccine. Age maintained the most powerful
predictor, with the oldest adults facing mortality rates more than 190
times higher than children, similar to the Poisson model.

\begin{figure}

\begin{minipage}{0.50\linewidth}
~~~~The negative binomial regression model had a much lower AIC value of
44,747.3 compared to that of the Poisson Regression model (480,266.0).
This demonstrates that the Negative Binomial approach exhibits a lower
estimated prediction error and therefore better fits the mortality data.
Despite this, the Negative Binomial model both under and overpredicted
covid deaths. The fan-shaped pattern indicates that the model's
prediction accuracy lessened as the number of deaths increased.
Regardless, the model does a decent job of predicting the number of
COVID deaths when the observed deaths are below about
1,500.\end{minipage}%
%
\begin{minipage}{0.50\linewidth}
\pandocbounded{\includegraphics[keepaspectratio]{../images/nb_performance.png}}\end{minipage}%

\end{figure}%

~~~~Finally, we plotted the interaction between Age Group and Sex across
ages, as well as the mortality rates by region in a panel plot style.
Our interaction demonstrated that after age 44, sex had a stronger
effect on predicted COVID-19 deaths, given the average COVID-19 deaths
for males is much higher than for females. After age 84, the effect of
sex on reverses, indicating an interaction between the two predictive
variables. One possible explanation for this change is that the aversge
lifespan for females in the data may have been longer. Secondly, our
panel plot confirmed that mortality rates were highest in all regions,
with Southern states being the highest and Northeastern states being the
lowest. This reflects possible differences in health policy and vaccine
administration.

\begin{figure}[H]

\begin{minipage}{0.50\linewidth}
\pandocbounded{\includegraphics[keepaspectratio]{../images/interaction.png}}\end{minipage}%
%
\begin{minipage}{0.50\linewidth}
\pandocbounded{\includegraphics[keepaspectratio]{../images/panelPlot.png}}\end{minipage}%

\end{figure}%

\begin{center}
\text{\Large Discussion and Conclusion}
\end{center}

~~~~This study examined how COVID-19 mortality varied across demographic
groups, geographic regions, and time in the United States from 2020 to
2023. By combining the CDC's provisional death counts with state level
population estimates, we were able to evaluate death rates per 100,000
people and use regression modeling in addition to visualizations to
identify the factors strongly associated with COVID-19 mortality.
Ultimately, these findings consistantly showed that age group, sex,
region, and year played significant roles in shaping mortality patterns.

~~~~Mortality peaked nationally in late 2020 and early 2021, with a
sharp decline in mid-2021. Age was the strongest predictor of mortality,
as death rates grew as people got older. Additionally, males
consistantly showed to have higher mortality in comparison to females.
Regionally, the South showed elevated death rates, while regions such as
the Northeast had very high early mortality followed by lower rates in
later years. These trends are evident in the visualizations and further
backed by our statistical models.

~~~~Our Poisson model was able to capture the count based
characteristics of the data, but was also able to show its strong
overdispersion. The Negative Binomial model was the best fit for the
data, showing that mortality decreased significantly over time, males
had higher death rates compared to females, and regional differences
were statisically significant. The Two-Way ANOVA further revealed that
the effect of age group on mortality changed based on males and females,
emphasizing the significance of demographic interactions.

~~~~All of these findings have meaningful implications for public
health, for example, the strong age related disparities emphasize the
importance of ensuring that older adults are protected through the aid
of vaccines, boosters, and early treatment access. Being able to
understand these differences can help create interventions for potential
future public health emergencies.

~~~~However, our project study does have its limitations. Firstly, our
dataset being count data made regression hard more difficult and less
insightful. Secondly, observations in our data are not statistically
independent, so standard models such as linear regression may have
uinderestimated uncertainty. In addition, the CDC intentionally
suppresses values for smaller populations to maintain confidentiality,
which left us with many missing values that we could not work with.
Finally, identifying true statistical significance was almost impossible
because our dataset has so many observations, giving each coefficient a
small p-value by nature.

~~~~In the future, we would like to use additional predictors like
vaccination coverage, variant spread, or county-level data to better
understand patterns. We would also like to use more advanced modeling
approaches, for example, time series models like a Generalized Additive
Model to better capture seasonal patters.

~~~~In conclusion, our analysis revealed a clear and statistical
understanding of how COVID-19 mortality varied across demographic and
geographic dimensions in the United States, providing insights that may
help implement preventative or early intervention health strategies in
the future.

\newpage

\begin{center}
\textbf{\Large References}
\end{center}

\vspace{0.5em}

\begin{enumerate}
\def\labelenumi{\arabic{enumi}.}
\item
  Arel-Bundock, V. (2022). modelsummary: Data and model summaries in R.
  \emph{Journal of Statistical Software}, \emph{103}(1), 1--23.
  https://doi.org/10.18637/jss.v103.i01
\item
  Becker, R. A., Wilks, A. R., Brownrigg, R., Minka, T. P., \& Deckmyn,
  A. (2023). maps: Draw geographical maps. R package version 3.4.2.
  https://CRAN.R-project.org/package=maps
\item
  Centers for Disease Control and Prevention. (2020). Provisional
  COVID-19 deaths by sex and age. National Center for Health Statistics.
  https://data.cdc.gov/
\item
  Centers for Disease Control and Prevention. (2025, June 11). Benefits
  of getting vaccinated.
  https://www.cdc.gov/covid/vaccines/benefits.html
\item
  Gohel, D., \& Skintzos, P. (2024). \emph{flextable: Functions for
  tabular reporting} (R package version 0.9.4).
  https://CRAN.R-project.org/package=flextable
\item
  Grolemund, G., \& Wickham, H. (2011). Dates and times made easy with
  lubridate. \emph{Journal of Statistical Software}, 40(3), 1--25.
  https://doi.org/10.18637/jss.v040.i03
\item
  Kassambara, A. (2023). \emph{ggpubr: `ggplot2' based publication ready
  plots} (R package version 0.6.0).
  https://CRAN.R-project.org/package=ggpubr
\item
  Pedersen, T. L. (2024). \emph{patchwork: The composer of plots} (R
  package version 1.2.0). https://CRAN.R-project.org/package=patchwork
\item
  Robinson, D., Hayes, A., \& Couch, S. (2024). broom: Convert
  statistical analysis objects into tidy tibbles. R package version
  1.0.5. https://CRAN.R-project.org/package=broom
\item
  United States Census Bureau. (2023). State population totals:
  2020--2023.
  https://www.census.gov/data/tables/time-series/demo/popest/2020s-state-total.html
\item
  Venables, W. N., \& Ripley, B. D. (2002). \emph{Modern applied
  statistics with S} (4th ed.). Springer.
\item
  Wickham, H. (2016). \emph{ggplot2: Elegant graphics for data analysis.
  Springer}.
\item
  Wickham, H., Averick, M., Bryan, J., Chang, W., McGowan, L. D.,
  François, R., \ldots{} \& Yutani, H. (2019). Welcome to the tidyverse.
  \emph{Journal of Open Source Software}, 4(43), 1686.
  https://doi.org/10.21105/joss.01686
\item
  Wickham, H., François, R., Henry, L., Müller, K., \& Vaughan, D.
  (2023). *dplyr: A grammar of data
\end{enumerate}




\end{document}
