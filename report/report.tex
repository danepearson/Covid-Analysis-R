% Options for packages loaded elsewhere
% Options for packages loaded elsewhere
\PassOptionsToPackage{unicode}{hyperref}
\PassOptionsToPackage{hyphens}{url}
\PassOptionsToPackage{dvipsnames,svgnames,x11names}{xcolor}
%
\documentclass[
]{article}
\usepackage{xcolor}
\usepackage{amsmath,amssymb}
\setcounter{secnumdepth}{5}
\usepackage{iftex}
\ifPDFTeX
  \usepackage[T1]{fontenc}
  \usepackage[utf8]{inputenc}
  \usepackage{textcomp} % provide euro and other symbols
\else % if luatex or xetex
  \usepackage{unicode-math} % this also loads fontspec
  \defaultfontfeatures{Scale=MatchLowercase}
  \defaultfontfeatures[\rmfamily]{Ligatures=TeX,Scale=1}
\fi
\usepackage{lmodern}
\ifPDFTeX\else
  % xetex/luatex font selection
\fi
% Use upquote if available, for straight quotes in verbatim environments
\IfFileExists{upquote.sty}{\usepackage{upquote}}{}
\IfFileExists{microtype.sty}{% use microtype if available
  \usepackage[]{microtype}
  \UseMicrotypeSet[protrusion]{basicmath} % disable protrusion for tt fonts
}{}
\makeatletter
\@ifundefined{KOMAClassName}{% if non-KOMA class
  \IfFileExists{parskip.sty}{%
    \usepackage{parskip}
  }{% else
    \setlength{\parindent}{0pt}
    \setlength{\parskip}{6pt plus 2pt minus 1pt}}
}{% if KOMA class
  \KOMAoptions{parskip=half}}
\makeatother
% Make \paragraph and \subparagraph free-standing
\makeatletter
\ifx\paragraph\undefined\else
  \let\oldparagraph\paragraph
  \renewcommand{\paragraph}{
    \@ifstar
      \xxxParagraphStar
      \xxxParagraphNoStar
  }
  \newcommand{\xxxParagraphStar}[1]{\oldparagraph*{#1}\mbox{}}
  \newcommand{\xxxParagraphNoStar}[1]{\oldparagraph{#1}\mbox{}}
\fi
\ifx\subparagraph\undefined\else
  \let\oldsubparagraph\subparagraph
  \renewcommand{\subparagraph}{
    \@ifstar
      \xxxSubParagraphStar
      \xxxSubParagraphNoStar
  }
  \newcommand{\xxxSubParagraphStar}[1]{\oldsubparagraph*{#1}\mbox{}}
  \newcommand{\xxxSubParagraphNoStar}[1]{\oldsubparagraph{#1}\mbox{}}
\fi
\makeatother


\usepackage{longtable,booktabs,array}
\usepackage{calc} % for calculating minipage widths
% Correct order of tables after \paragraph or \subparagraph
\usepackage{etoolbox}
\makeatletter
\patchcmd\longtable{\par}{\if@noskipsec\mbox{}\fi\par}{}{}
\makeatother
% Allow footnotes in longtable head/foot
\IfFileExists{footnotehyper.sty}{\usepackage{footnotehyper}}{\usepackage{footnote}}
\makesavenoteenv{longtable}
\usepackage{graphicx}
\makeatletter
\newsavebox\pandoc@box
\newcommand*\pandocbounded[1]{% scales image to fit in text height/width
  \sbox\pandoc@box{#1}%
  \Gscale@div\@tempa{\textheight}{\dimexpr\ht\pandoc@box+\dp\pandoc@box\relax}%
  \Gscale@div\@tempb{\linewidth}{\wd\pandoc@box}%
  \ifdim\@tempb\p@<\@tempa\p@\let\@tempa\@tempb\fi% select the smaller of both
  \ifdim\@tempa\p@<\p@\scalebox{\@tempa}{\usebox\pandoc@box}%
  \else\usebox{\pandoc@box}%
  \fi%
}
% Set default figure placement to htbp
\def\fps@figure{htbp}
\makeatother





\setlength{\emergencystretch}{3em} % prevent overfull lines

\providecommand{\tightlist}{%
  \setlength{\itemsep}{0pt}\setlength{\parskip}{0pt}}



 


\makeatletter
\@ifpackageloaded{caption}{}{\usepackage{caption}}
\AtBeginDocument{%
\ifdefined\contentsname
  \renewcommand*\contentsname{Table of contents}
\else
  \newcommand\contentsname{Table of contents}
\fi
\ifdefined\listfigurename
  \renewcommand*\listfigurename{List of Figures}
\else
  \newcommand\listfigurename{List of Figures}
\fi
\ifdefined\listtablename
  \renewcommand*\listtablename{List of Tables}
\else
  \newcommand\listtablename{List of Tables}
\fi
\ifdefined\figurename
  \renewcommand*\figurename{Figure}
\else
  \newcommand\figurename{Figure}
\fi
\ifdefined\tablename
  \renewcommand*\tablename{Table}
\else
  \newcommand\tablename{Table}
\fi
}
\@ifpackageloaded{float}{}{\usepackage{float}}
\floatstyle{ruled}
\@ifundefined{c@chapter}{\newfloat{codelisting}{h}{lop}}{\newfloat{codelisting}{h}{lop}[chapter]}
\floatname{codelisting}{Listing}
\newcommand*\listoflistings{\listof{codelisting}{List of Listings}}
\makeatother
\makeatletter
\makeatother
\makeatletter
\@ifpackageloaded{caption}{}{\usepackage{caption}}
\@ifpackageloaded{subcaption}{}{\usepackage{subcaption}}
\makeatother
\usepackage{bookmark}
\IfFileExists{xurl.sty}{\usepackage{xurl}}{} % add URL line breaks if available
\urlstyle{same}
\hypersetup{
  pdftitle={COVID-19 Mortality Anaysis Report},
  pdfauthor={Dane Pearson; Dhriti Avala},
  colorlinks=true,
  linkcolor={blue},
  filecolor={Maroon},
  citecolor={Blue},
  urlcolor={Blue},
  pdfcreator={LaTeX via pandoc}}


\title{COVID-19 Mortality Anaysis Report}
\usepackage{etoolbox}
\makeatletter
\providecommand{\subtitle}[1]{% add subtitle to \maketitle
  \apptocmd{\@title}{\par {\large #1 \par}}{}{}
}
\makeatother
\subtitle{A Statistical Investigation using R}
\author{Dane Pearson \and Dhriti Avala}
\date{2025-12-05}
\begin{document}
\maketitle


\section{MAXIMUM 6 PAGES !}\label{maximum-6-pages}

\section{Introduction (0.75 - 1 page)}\label{introduction-0.75---1-page}

\begin{itemize}
\tightlist
\item
  Research question or paper overview
\item
  Background and motivation (make sure to include proper citations)
\item
  Dataset description (source, size, key variables)
\item
  Study objectives
\end{itemize}

\section{Exploratory Data Analysis (1 - 1.5
pages)}\label{exploratory-data-analysis-1---1.5-pages}

\begin{itemize}
\tightlist
\item
  Summary statistics
\item
  Data visualization (2-3 key plots)
\item
  Data quality assessment (missing values, outliers, distributions)
\item
  Preliminary insights that inform modeling choices
\end{itemize}

\section{Methods (1 - 1.5 pages)}\label{methods-1---1.5-pages}

\begin{itemize}
\tightlist
\item
  Statistical models used, with mathematical notation where appropriate

  \begin{itemize}
  \tightlist
  \item
    \textbf{For regression:} specify model equation, link function (if
    applicable), assumptions
  \item
    \textbf{For PCA:} explain dimensionality reduction approach
  \item
    \textbf{For clustering:} describe method and distance metric
  \item
    \textbf{For regularization:} specify penalty type and selection
    procedure
  \end{itemize}
\item
  Why these methods? Connect to research questions and structure of data
\item
  Software and key R packages used (don't forget to \textbf{\emph{cite}}
  the R packages)
\end{itemize}

\section{Results (2 - 2.5 pages)}\label{results-2---2.5-pages}

\begin{itemize}
\tightlist
\item
  Model fitting and diagnostics
\item
  Assumption checking (residual plots, normality tests, etc.)
\item
  Model comparison (if applicable)
\item
  Goodness-of-fit measures
\item
  Parameter interpretation with confidence intervals where appropriate
\item
  Key findings presented with visualizations
\end{itemize}

\section{Discussion \& Conclusion (0.75 - 1
page)}\label{discussion-conclusion-0.75---1-page}

\begin{itemize}
\tightlist
\item
  Answers to research questions with supporting evidence
\item
  Practical implications or insights
\item
  Limitations and assumptions
\item
  Future directions
\end{itemize}

\section{References (not included in page
count)}\label{references-not-included-in-page-count}

\begin{itemize}
\tightlist
\item
  Use consistent citation style (APA, Chicago, or similar)
\item
  Include at least 2-3 references beyond the dataset source
\end{itemize}




\end{document}
